\begin{Abstract}
%Advancement in photovoltaics as an alternative energy resource is restricted due to the limitations in the manufacturing process of thin silicon wafers. Traditional abrasive methods lead to higher kerf (material) losses. Application of micro-drilling process of Wire Electro Discharge Machining (WEDM) is a viable alternative to such traditional methods. However, WEDM is not as well established for cutting of semiconductors as it is for metals due to the lack of electrical characterisation of metal-dielectric-semiconductor conduction.

Dissemination of photovoltaics is limited due to large payback period and huge capital costs. Cost of raw silicon and that of its fabrication contribute major expenses during initial stages of photovoltaics manufacturing chain. Cost of fabrication of silicon wafers can be reduced by alternative manufacturing processes such as Wire Electric Discharge Machining.

This report summarises the work directed towards the fabrication of an indigenous power supply unit for such unconventional load investigations. The pulsed power supply for this purpose is designed in this work. A parallel converter topology comprising of two modified DC-DC converters is described.

The procedure to derive the linear and nonlinear state space model of this modified converter is presented. Three linear control strategies viz. PID, current mode control, and compensator based control have been tried and the simulation results are included for this converter. The hardware setup employing linear control has been developed for this purpose. The results of initial experimentation are also presented.

The sliding mode control technique is reviewed and a modified version is simulated to verify mitigation of steady state error.
\end{Abstract}